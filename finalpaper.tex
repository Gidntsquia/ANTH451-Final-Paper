\documentclass[a4paper, 12pt]{article}

% Packages
\usepackage[utf8]{inputenc} % UTF-8 encoding
\usepackage[T1]{fontenc}    % Font encoding
\usepackage[english]{babel}  % Language
\usepackage{graphicx}       % For including images
\usepackage{amsmath}        % Math packages
\usepackage{amsfonts}
\usepackage{amssymb}
\usepackage{natbib}         % Bibliography package
\usepackage{url}            % URL formatting
\usepackage{hyperref}       % Hyperlinks
\usepackage{geometry}       % Adjust page margins
\usepackage{setspace}       % Adjust line spacing

% Title and Author
\title{Archaeological Exploration: A Journey into the Past}
\author{Jaxon Lee}
\date{\today}

\begin{document}

\maketitle

\begin{abstract}
    % Your abstract goes here.
    This is a brief summary of your paper.
\end{abstract}

\section{Introduction}

% Your introduction section goes here.
\cite{johnson2005}.

\section{Strontium Analysis}

% What is it? How does it work? History of it
% What is it
Strontium isotope analysis is the analysis of Strontium (Sr) isotope ratios to
understand geographic movement of humans and animals.
- "Provenance" - place of origin

% % How it works
Here is how it works.
% Main premise
- Each region of rocks has a unique 87Sr / 86Sr ratio
- Plants and animals inherit the 87Sr/86Sr ratio of their environment (isoscape)
- If we know the 87Sr/86Sr ratio of a region and the 87Sr/86ASr ratio of organic matter, we can tell if that organic matter came from that region.

"Strontium has four naturally occurring isotopes: 88Sr, 87Sr, 86Sr, and 84Sr. 87Sr is formed as the radiogenic daughter isotope of 87Rb (rubidium); the decay of 87Rb leads to different abundances of 87Sr in rocks depending on their age and their original 87Rb content (Dickin, 1995). The ratio of the radiogenic 87Sr to the naturally abundant 86Sr is variable across lithologies of different ages and with different formation histories. Due to the 48.8 billion year half-life of 87Rb (Faure and Mensing, 2005, p. 77), the ratio of 87Sr to 86Sr does not change significantly over the time scales that are of interest to researchers in archaeology, biology, forensics, food science, and other disciplines that deal with the comparatively recent past. This relative stability of the 87Sr/86Sr ratio allows strontium isotopes to be used to provenance biological materials that have taken up strontium from their environments."
% https://www.sciencedirect.com/science/article/pii/S0012825221000933?casa_token=FLhhJysXAz4AAAAA:VTVcEUD71ZsaI0siB8F-2VvDoI03s2EC1zN-20NrsRFemTiffkhW5X0yz_8Iv1-NxARwqwhL6vE

% How they make isoscapes
- Domain mapping, contour mapping, machine learning
- Sometimes predict
- Collect data for Sr ratios in regions today
- Combine them using various methods, such as "random forest regression method"
% https://www.sciencedirect.com/science/article/abs/pii/S0031018220302947


\section{Major Areas Interrogated}

% Describe the methods you used in your research.

\section{Specific Case Studies}

% Present your archaeological findings and discoveries.


\section{Conclusion}

% Summarize the key points and conclusions of your paper.

\section*{Acknowledgments}

% Acknowledge any individuals or organizations that contributed to your research.

% Bibliography
\bibliographystyle{plainnat}
\bibliography{references} % Specify your bibliography file (e.g., references.bib)

\end{document}
