\documentclass[a4paper, 12pt]{article}

% Packages
\usepackage[utf8]{inputenc} % UTF-8 encoding
\usepackage[T1]{fontenc}    % Font encoding
\usepackage[english]{babel}  % Language
\usepackage{graphicx}       % For including images
\usepackage{amsmath}        % Math packages
\usepackage{amsfonts}
\usepackage{amssymb}
\usepackage{natbib}         % Bibliography package
\usepackage{url}            % URL formatting
\usepackage{hyperref}       % Hyperlinks
\usepackage{geometry}       % Adjust page margins
\usepackage{setspace}       % Adjust line spacing

% Title and Author
\title{Survey of Strontium Isotope Analysis in Archeolgical Research of Ancient Egypt}
\author{Jaxon Lee}
\date{\today}

\begin{document}

\maketitle

\begin{abstract}
    Archeologists often dig up human skeletal remains. One common tool for learning
    more about these is isotope analysis, which involves investigating the levels of various
    elements such as oxygen, carbon, or strontoum using chemistry. Strontium isotope analysis
    in particular is useful for archeologists since it helps them understand the geographic
    movement of humans and animals.
\end{abstract}

\section{Introduction}
Archeologists often dig up human skeletal remains. One common tool for learning
more about these is isotope analysis, which involves investigating the levels of various
elements such as oxygen, carbon, or strontoum using chemistry. Strontium isotope analysis
in particular is useful for archeologists since it helps them understand the geographic
movement of humans and animals.

In this paper, I will detail how strontium isotope analysis works, its main
use cases, and a few interesting case studies that utilize it.
% - Purpose
% - General use cases
% - Setting up what I'm going to say in this paper (i.e. detail how the method works, discuss main use cases, and delve into specific case studies)


\section{Strontium Analysis}
\subsection{Overview}
% What is it? How does it work? History of it

% What is it

\subsection{Rationale}
% Rationale for using Strontium 
Sr is an element, which occurs naturally at varying concentrations in rock formations.
Strontium gets into the water stream through erosion and eventually is inadvetently consumed by plants and animals in trace amounts \citep{bartelink2019}.
Eventually, when humans or animals inevitably consume plants, water, or other animals,
a small amount of strontium gets into their bones and tissue. Notably, although the amount is trace, the
ratio of strontium stays constant throughout all these processes since there is no "isotopic fractionation" \citep{bartelink2019}.
Thus, measuring strontium in bones or tissue gives a picture of where humans or animals source their food and water.
Measuring the strontium level of longer bones gives insight into the last 7-10
years of a person's life and measuring the strontium of hair can tell where someone
took residence immediately prior to death \citep{kamenov2014}

Although this is trivially already useful for fields such as forensics \citep{kamenov2014},
archeologists usually have a good idea of where a person lived before they died
since people are usually buried where they lived. However, since tooth enamel
forms during childhood and does not change, measuring it can give the general location
that the person lived in during their tooth formation, i.e., when they were a child
\citep*{holt2021,kozieradzkaogunmakin2021,lazzerini2021}. Thus, archeologists can
identify the "provenance," or place of origin, of skeletal remains they dig up \citep{holt2021}.


% Purpose
\subsection{Use Cases}
- "Provenance" - place of origin \citep{holt2021}
- Preminent goal is to track movement of animals and humans
- "Local vs non local" \citep{holt2021}
- Can be used on "ancient organisms" \citep{crowley2017}
- Origin of glasswork [11] -- "Alexandrian glass"

% % How it works
Here is how it works.
% Main premise
- Each region of rocks has a unique 87Sr / 86Sr ratio
- Plants and animals inherit the 87Sr/86Sr ratio of their environment (isoscape) [4]
- If we know the 87Sr/86Sr ratio of a region and the 87Sr/86ASr ratio of organic matter, we can tell if that organic matter came from that region.
--> No "fractionation" \citep{bartelink2019}
- For humans: sample tooth enamel b/c this is formed in childhood. \citep{kozieradzkaogunmakin2021} \citep{holt2021}
--> also \citep{lazzerini2021}
--> also doesn't really break down \citep{kozieradzkaogunmakin2021}
--> Enamel = childhood, Bones = last "7-10" years, Hair = immediately prior to death \citep{kamenov2014}

"Strontium has four naturally occurring isotopes: 88Sr, 87Sr, 86Sr, and 84Sr. 87Sr is formed as the radiogenic daughter isotope of 87Rb (rubidium); the decay of 87Rb leads to different abundances of 87Sr in rocks depending on their age and their original 87Rb content (Dickin, 1995). The ratio of the radiogenic 87Sr to the naturally abundant 86Sr is variable across lithologies of different ages and with different formation histories. Due to the 48.8 billion year half-life of 87Rb (Faure and Mensing, 2005, p. 77), the ratio of 87Sr to 86Sr does not change significantly over the time scales that are of interest to researchers in archaeology, biology, forensics, food science, and other disciplines that deal with the comparatively recent past. This relative stability of the 87Sr/86Sr ratio allows strontium isotopes to be used to provenance biological materials that have taken up strontium from their environments."
%  TODO: Go into more specific detail on how researchers carry this out.

% \citep{holt2021} https://www.sciencedirect.com/science/article/pii/S0012825221000933?casa_token=FLhhJysXAz4AAAAA:VTVcEUD71ZsaI0siB8F-2VvDoI03s2EC1zN-20NrsRFemTiffkhW5X0yz_8Iv1-NxARwqwhL6vE
% \citep{crowley2017} https://onlinelibrary.wiley.com/doi/full/10.1111/brv.12217?casa_token=fzF_HvyFA3EAAAAA%3ACWLmQaH1S_B4GXOgCmSZJYwQr-NNwC4iUxP1kPFA8Tw05-d65rejwmCkDgQ3itV6APu8LbHSWdOOcuCY
% [4] https://link.springer.com/article/10.1007/BF01100444
% \citep{kozieradzkaogunmakin2021} https://brill.com/display/book/9789004433755/BP000007.xml
% [11] https://www.nature.com/articles/s41598-020-68089-w
% \citep{kamenov2014} https://anthrosource.onlinelibrary.wiley.com/doi/full/10.1111/napa.12048?casa_token=rWbuuiJYOvIAAAAA%3AcVqQVNRLjm3TVnsmeJyJ4JxPgPuKJeEbIRmWrQ_lV49lztX-XaL17cd9bn3bxJeketNDv0zj1b2jcWRT
% \citep{lazzerini2021} https://www.nature.com/articles/s41598-021-81923-z

\subsection{Isoscapes}
% How they make isoscapes
In order for strontium measuremnts to be useful, archeologists need a baseline to compare to.
So, much of the research into strontium isotope analysis in the last decade has gone
into mapping "isoscapes," which are maps of the expected strontium isotope ratios of tissue in various geographic regions (CITATION NEEDED).
]
I will discuss the three main approaches for creating an isoscape: domain mapping,
contour mapping, and machine learning \citep{bataille2020}.
\subsubsection{Domain Mapping}
\subsubsection{Contour Mapping}
\subsubsection{Machine Learning}



- Domain mapping, contour mapping, machine learning \citep{bataille2020}
- Sometimes predict \citep{bataille2020}
- Collect data for Sr ratios in regions today \citep{bataille2020} [8, section 2.2]
- Combine them using various methods, such as "random forest regression method" \citep{bataille2020}
- Isoscape size can be large, which can make location precision low \citep{holt2021}

% \citep{bataille2020} https://www.sciencedirect.com/science/article/abs/pii/S0031018220302947 
% [10] http://eprints.bournemouth.ac.uk/33461/ -- example of mapping isoscape using animals

% History
- Originally about rock erosion and where Sr came from regarding rivers \citep{crowley2017}
- Late 80s/early 90s-- first uses of Sr to detect where person came from. Lots of studies to prove that it could viably be used in this way\citep{crowley2017}
- Ramping up in last decade \citep{crowley2017}
- A lot of recent advancements  \citep{holt2021}
-> "high performance laser ablation" and "multicollector inductively coupled plasma mass spectrometry"
- Current goal- isoscapes \citep{holt2021}


% Limitations
- Often is not precise to conclusively answer questions by itself. Best used in combination with other methods, such as analyzing other isotopes. \citep{holt2021}
- Expensive \citep{crowley2017}

% Sources:
% [Same 1] https://www.sciencedirect.com/science/article/pii/S0012825221000933?casa_token=H3KiKu8CjckAAAAA:B4tj30AghfGVDqr1qDlsstbfhMKHja2hn4i0s8Iopsn4oVXcOvlsbJpViADpjDOJlWZUdAWuRXo


\section{Main Areas}
- Ancient habitat use \citep{crowley2017}
- Animal origins \citep{crowley2017}
--> "anadromous fish", "extinct hominins"
- Farm product sources, like rice, corn, and drugs \citep{crowley2017}
- Migration routes \citep{crowley2017}
- Where illegally poached animals came from \citep{crowley2017}
- Range of invasive species \citep{crowley2017}
- Understand landscape use \citep{crowley2017}
- Forensics \citep{kamenov2014}

I've chosen Ancient Egypt because Sr analysis is particularly easy to apply b/c of mummies.
- Lots of stuff on the Hyksos. [5] [7] [9] [12]
- \citep{kozieradzkaogunmakin2021} -- combining Sr analysis with diet analysis to see dietary differences b/w locals and non-locals. No meaningful difference found. Conclusion: confirming previous diet research, can't really use Oxygen analysis to determine local-ness

% [Same 1] https://www.sciencedirect.com/science/article/pii/S0012825221000933?casa_token=H3KiKu8CjckAAAAA:B4tj30AghfGVDqr1qDlsstbfhMKHja2hn4i0s8Iopsn4oVXcOvlsbJpViADpjDOJlWZUdAWuRXo
% [7] https://link.springer.com/article/10.1007/s12520-021-01344-x
% [9] http://eprints.bournemouth.ac.uk/33312/1/17_Stantis_CAENL%209.pdf -- importance of collecting modern samples, combining Sr with other isotope analyis, and building on previous mobility work
% [12] http://eprints.bournemouth.ac.uk/33511/1/N.%20Maaranen1_03.10.19.pdf
% \citep{bartelink2019} https://academic.oup.com/fsr/article/4/1/29/6794654

\section{Specific Case Studies}
\subsection{Hyksos}
% Hyksos
- 1638 BC - 1530 BC [5]
- Foreign Hysos rise to power
- Lots of non-local women before Hyksos rule - likely gradual power grab by Hyksos, which contradicts historical narrative [5]

- Original narrative: Egyptian priest Manetho-- terrible invasion. But, this was a biased source, albeit the only available source. [5]
- Methods:
--> Excavating various graves
--> Check tooth enamel (which was formed during childhood) for falling in "local" range of Sr values
- Conclusions:
--> Non-local people came from all over
--> Hyksos were not an invading source. They arrived centuries before and gradually rose to power.
% [5] https://journals.plos.org/plosone/article?id=10.1371/journal.pone.0235414

\subsection{Mummified Birds}
% Mummified Birds
- Where ancient Egyptian mummified birds came from
- Farmed vs hunted -- capabilities, economy, and effect on environment
- Some bird gods like Horus and Thoth
- Methods:
--> Take bone samples from birds
--> Combination of lots of isotope analyses, including Sr
- Results:
--> 8/11 ibises local, birds of prey were not local
- Conclusions:
--> ibises and birds of prey were wild-- they all moved a good deal (ibises a little, birds of prey a lot)
% [6] https://www.nature.com/articles/s41598-020-72326-7

\subsection{Migrational Origins in Ancient Egypt}
% Migrational Origins in Ancient Egypt
- 2500 BCE - 656 BCE
- Figuring out mobility of rural and urban settlements in Egypt over time
- Methods:
--> Dental Sr to determine localness of Abu Fatima, Hannek, and Tombos graves
- Conclusions:
--> Across the board, there were some non-local people, which indicates that migration was normal b/w First and Second Nile Cataract
--> Even poorer people migrated
% [13] https://www.sciencedirect.com/science/article/pii/S2352409X18305509?casa_token=5XDdWMPYLpEAAAAA:YAvhcSklHzL7SZB_V2M3eqXaZK8AzFKP4Iw3pTXD0Opp1u-KSGONBW4X3lEHp5sc62fgnhWs0oU#s0015

\section{Conclusion}
-

\section*{Acknowledgments}

% Acknowledge any individuals or organizations that contributed to your research.

% Bibliography
\bibliographystyle{apalike}
\bibliography{references} % Specify your bibliography file (e.g., references.bib)

\end{document}
